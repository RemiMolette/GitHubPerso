% This LaTeX template is intended for the students of CSI Master from
% University of Bordeaux to make their reports.
%
% This template can be used and modified with no restriction.
%
% %%% History %%%
%
% * April 22, 2014: First version (Emmanuel Fleury <fleury@labri.fr>)
% * May 09, 2016    First realase (Rémi Tremblain <remi.tremblain@soprasteria.com>)
%
% 
%

\documentclass[a4paper]{memoir}

%%%%% Packages %%%%%
\usepackage{lmodern}
\usepackage{palatino}
\usepackage[T1]{fontenc}
\usepackage[utf8]{inputenc}

% To be removed if you want it in english
\usepackage[french]{babel}
 

\usepackage{amstext,amsmath,amssymb,amsfonts}
\usepackage{multirow,colortbl}
\usepackage{xspace,varioref}
\usepackage{hyperref}

\usepackage[dvipsnames]{xcolor}
\usepackage{graphicx}

\usepackage{appendix}
\usepackage{makeidx}

%% custom style %%%%%%%%%%%%%%%%%%%%%%%%%%%%%%%%%%%%%%%%%%%%%%%%%%%%%%%%

% custom commands
\newcommand{\version}[1]{\def\theversion{#1}}
\newcommand{\subtitle}[1]{\def\thesubtitle{#1}}

\newcommand{\authors}[1]{\def\theauthors{#1}\author{#1}}
\newcommand{\supervisor}[1]{\def\thesupervisor{#1}}
\newcommand{\tutor}[1]{\def\thetutor{#1}}

% translation for custom words
\newcommand{\authorname}{Author}
\newcommand{\authorsname}{Authors}
\newcommand{\supervisorname}{Supervisor}
\newcommand{\tutorname}{Tutor}

\newcommand{\thepartname}{Part}

\ifdefined\addto{%
\addto{\captionsfrench}{\renewcommand{\authorname}{Auteur}}%
\addto{\captionsfrench}{\renewcommand{\authorsname}{Auteurs}}%
\addto{\captionsfrench}{\renewcommand{\supervisorname}{Superviseur}}%
\addto{\captionsfrench}{\renewcommand{\tutorname}{Tuteur}}}
\addto{\captionsfrench}{\renewcommand{\thepartname}{Partie}}
\else{}
\fi

%%%%% Setting Titlepage %%%%%
%%%%%%%%%%%%%%%%%%%%%%%%%%%%%
\pretitle{\flushleft\Huge\textsf}
\posttitle{\\[-.65em]\rule{\linewidth}{1.5mm}\\[-.65em]
\ifx\thesubtitle\undefined%
\else%
  \hfill{\small\itshape \thesubtitle}%
\fi
\centering
\vfill
\includegraphics{Universite_de_Bordeaux.pdf}
\vfill
\iflanguage{french}{%
  {\Huge\bfseries Mémoire de fin d'étude}
% {\Huge Projet de deuxième année}
% {\Huge Projet de première année}
}{%
  {\Huge Master Thesis}
% {\Huge Master2 Project}
% {\Huge Master1 Project}
}\\
\vspace{1.25em}
\iflanguage{french}{%
  \LARGE
  Master \emph{Sciences et Technologies},\\
  Mention \emph{Informatique},\\
% Mention \emph{Mathématiques},\\
  Parcours \emph{Cryptologie et Sécurité Informatique}.\\
  \par\hfill%
}{%
  \LARGE
  Master in \emph{Sciences and Technologies},\\
  Specialty in \emph{Computer Science},\\
% Specialty in \emph{Mathematics},\\
  Track \emph{Cryptology and Computer Security}.\\
  \par\hfill
}}

%% author
\preauthor{\vspace{\fill}\\
\ifx\theauthors\undefined%
  \flushleft\textbf{\large\authorname}\\
\else%
  \flushleft\textbf{\large\authorsname}\\
\fi
\small}
\postauthor{\vspace{1em}
\ifx\thesupervisor\undefined%
\else%
  \newline\textbf{\large\supervisorname}\\\thesupervisor\\[1em]%
\fi
\ifx\thetutor\undefined%
\else%
  \textbf{\large\tutorname}\\\thetutor%
\fi
\\[-.25em]
\rule{\linewidth}{1mm}\\[-.25em]}

%% version and date
\predate{\hspace{\fill}
\ifx\theversion\undefined%
\else%
  version~\theversion~--~%
\fi}
\postdate{}

%% chapters style %%%%%%%%%%%%%%%%%%%%%%%%%%%%%%%%%%%%%%%%%%%%%%%%%%%%%%
%% You may try several styles (see more in the memoir manual).

\chapterstyle{veelo}
%\chapterstyle{chappell}
%\chapterstyle{ell}
%\chapterstyle{ger}
%\chapterstyle{pedersen}
%\chapterstyle{verville}
%\chapterstyle{madsen}
%\chapterstyle{thatcher}

%% parts style %%%%%%%%%%%%%%%%%%%%%%%%%%%%%%%%%%%%%%%%%%%%%%%%%%%%%%%%%

\renewcommand*{\thepart}{\arabic{part}}

\renewcommand*{\parttitlefont}{\chaptitlefont\Huge}
\renewcommand*{\partnamefont}{\chapnamefont\HUGE}
\renewcommand*{\partnumfont}{\chapnumfont\HUGE}

\renewcommand{\beforepartskip}{\vspace*{\fill}}
\renewcommand{\midpartskip}{\vspace{.5em}\hrule height 1.5mm \vspace{.5em}}
\renewcommand{\afterpartskip}{\vspace*{\fill}}

% table of contents
\renewcommand*{\cftpartname}{\thepartname}
\renewcommand*{\cftpartpresnum}{\space}
\renewcommand*{\cftpartaftersnum}{.}
\renewcommand*{\cftpartaftersnumb}{\space}

\cftpagenumbersoff{part}
\renewcommand{\cftpartafterpnum}{\protect\\[-.75em]%
  \protect\mbox{}\protect\hrule\par}

\renewcommand{\cftchapterdotsep}{4}


%% index generation %%%%%%%%%%%%%%%%%%%%%%%%%%%%%%%%%%%%%%%%%%%%%%%%%%%%
\makeindex

%%%%% Useful macros %%%%%
\newcommand{\latinloc}[1]{\ifx\undefined\lncs\relax\emph{#1}\else\textrm{#1}\fi\xspace}
\newcommand{\etc}{\latinloc{etc}}
\newcommand{\eg}{\latinloc{e.g.}}
\newcommand{\ie}{\latinloc{i.e.}}
\newcommand{\st}{\ensuremath{\text{\xspace s.t.\xspace}}}


%%%%% Report Title %%%%%
\title{Audit technique et pentest}
\subtitle{Etat de l'art, situation et évolution}

% If only one author use \author
\author{Rémi Tremblain \texttt{<remi.tremblain@etu.u-bordeaux.fr>}}

% If several authors use \authors{}
%\authors{Jean Dupont \texttt{<jean.dupont@etu.u-bordeaux.fr>}\\
%Stéphanie Martin \texttt{<stephanie.martin@etu.u-bordeaux.fr>}}

\supervisor{Pierrick Conord \texttt{<pierrick.conord@soprasteria.com>}}

\tutor{Emmanuel Fleury \texttt{<fleury@labri.fr>}}

%\version{0.1}

%%%%% Document %%%%%
%%%%%%%%%%%%%%%%%%%%
\begin{document}

\frontmatter%%%%%%%%%%%%%%%%%%%%%%%%%%%%%%%%%%%%%%%%%%%%%%%%%%%%%%%%%%%%
\maketitle
\thispagestyle{empty}

\input{declaration}

\chapter*{Résumé}
%\chapter*{Abstract}

\index{Résumé} Le mone de la sécurité informatique est en constante évolution et nous voyons aujourd'hui l'avancement de ce domaine.

Axe de dév

Domaine de plus en plus important, de plus en plus prisé et reconnu
important dans n'importe qu'elle infrastructure 
Forte croissance du nombre de machine et de la demande en sécurité informatique derriere 
D'où vient cette necessité ? Qu'en est-il de maintenant ? Où va-t-on ? le modèle de bug bounty ba bla


\chapter*{Présentation de l'entreprise}

Ce travail a été effectué dans le cadre du stage de fin d'étude qui se déroulait sur une période de six mois, au sein de la société Sopra Steria.
Les informations qui suivent sont une présentation de la société, de son organisation ainsi que son activité.

\section{Activité}

\textbf{A retravailler}\\

Sopra Steria est un leader européen de la transformation numérique. Le groupe se positionne dans le top quatre en France et dans le top dix en Europe des soéciétés de services IT (source Gartner).\\
Avec un chiffre d'affaire annuel de 3,6 Milliards d'euros en 2015 et plus de 37 000 colaboraters Sopra steria est localisée dans plus de vingts pays à travers le monde, principalement en Europe.\\

La révolution du numérique est une source d'opportunités pour nos clients dans leur transformtion et la fourniture de service de qualité tout en maitrisant cet écosystème IT ainsi que son coût.\\
Sopra Steria se définit en quatres points :

\begin{itemize}
    \item Accompagner nos clients dans leur transformation numérique.
    \item Créer et opérer des services innovants pour faire de ces mutations un atout majeur.
    \item Gagner en réactivité et flexibilité pour accompagner la coissance et la compétitivité.
    \item Maîtriser la qualité et les coûts sur les systèmes existants.
\end{itemize}

A revoir

\cleardoublepage
\tableofcontents*

%%%%%%%%%%%%%%%%%%%%%%%%%%%%%%%%%%%%%%%%%%%%%%%%%%%%%%%%%%%%%%%%%%%%%%%%

\chapter*{Introduction}



%%%%%%%%%%%%%%%%%%%%%%%%%%%%%%%%%%%%%%%%%%%%%%%%%%%%%%%%%%%%%%%%%%%%%%%%

\mainmatter%%%%%%%%%%%%%%%%%%%%%%%%%%%%%%%%%%%%%%%%%%%%%%%%%%%%%%%%%%%%%
\part{Etat de l'art \& aspect théorique}

\chapter{Etat de l'art}

\chapter{Aspect théorique}

%%%%%%%%%%%%%%%%%%%%%%%%%%%%%%%%%%%%%%%%%%%%%%%%%%%%%%%%%%%%%%%%%%%%%%%%
\part{Situation}

\chapter{En terme de technique}

\chapter{En terme juridique}

\chapter{Constat, ce qu'il faut modifier, pour faire le lien avec la partie d'après}


%%%%%%%%%%%%%%%%%%%%%%%%%%%%%%%%%%%%%%%%%%%%%%%%%%%%%%%%%%%%%%%%%%%%%%%%
\part{Evolution}

\chapter{En terme de technique}

\chapter{Pour el grand public ?}

%%%%%%%%%%%%%%%%%%%%%%%%%%%%%%%%%%%%%%%%%%%%%%%%%%%%%%%%%%%%%%%%%%%%%%%%

\chapter*{Conclusion}

bla 

%%%%%%%%%%%%%%%%%%%%%%%%%%%%%%%%%%%%%%%%%%%%%%%%%%%%%%%%%%%%%%%%%%%%%%%%

\part*{Annexes}
\addcontentsline{toc}{part}{Annexes}
\appendix

\chapter{Aux cas où}

\chapter{Pour la forme}


\backmatter%%%%%%%%%%%%%%%%%%%%%%%%%%%%%%%%%%%%%%%%%%%%%%%%%%%%%%%%%%%%%

\nocite{*}
\bibliographystyle{plain}
\bibliography{bibliography}

\printindex

\end{document}


